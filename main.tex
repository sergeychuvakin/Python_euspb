\documentclass[a4paper,11pt]{article}

%pdflatex
\usepackage{cmap}					% поиск в PDF
\usepackage{mathtext} 				% русские буквы в формулах
\usepackage[T2A]{fontenc}			% кодировка
\usepackage[utf8]{inputenc}			% кодировка исходного текста
\usepackage[english, russian]{babel}	
\usepackage{indentfirst}
\frenchspacing

%математика
\usepackage{amsmath,amsfonts,amssymb,amsthm,mathtools} % AMS
\usepackage{icomma}
\newcommand*{\hm}[1]{#1\nobreak\discretionary{}
	{\hbox{$\mathsurround=0pt #1$}}{}}

\usepackage{extsizes} % Возможность сделать экстрашрифт
\usepackage{geometry} 
\geometry{top=30mm}
\geometry{bottom=40mm}
\geometry{left=30mm}
\geometry{right=20mm}

\usepackage{setspace}
\onehalfspacing

\usepackage{hyperref}
\usepackage[usenames,dvipsnames,svgnames,table,rgb]{xcolor}
\hypersetup{				% Гиперссылки
	unicode=true,           % русские буквы в раздела PDF
	pdftitle={Заголовок},   % Заголовок
	pdfauthor={Автор},      % Автор
	pdfsubject={Тема},      % Тема
	pdfcreator={Создатель}, % Создатель
	pdfproducer={Производитель}, % Производитель
	pdfkeywords={keyword1} {key2} {key3}, % Ключевые слова
	colorlinks=true,       	% false: ссылки в рамках; true: цветные ссылки
	linkcolor=black,          % внутренние ссылки
	citecolor=black,        % на библиографию
	filecolor=magenta,      % на файлы
	urlcolor=blue           % на URL
}
\author{Чувакин С., Румянцева А.}
\title{Python как язык программирования для анализа данных и разработки}
\date{\today}

%\let\endtitlepage\relax % убрать pagebreak после titlepage
\usepackage{enumitem} % цветные itemize (xcolor)
\setcounter{secnumdepth}{0} % убрать нумерацию секций

\usepackage{csquotes}



\begin{document}

\renewcommand{\abstractname}{Концепция} % назвать абстракт по другому

\maketitle

\begin{abstract}
     Курс расчитан на один семестр - 14 недель. Аудитория - слушатели не знакомые с программированием или знакомые на самом начальном уровне. Главная цель курса не только познакомить с языком, но и дать практический опыт. Курс предполагает затронуть аспекты связанные с анализом и обработкой данных, а также темы связанные с разработкой. Формат предполагает встреч раз в неделю на сдвоенную пару. 56 часов всего. По окончании слушатели будут уметь: читать python код, писать свой собсвенный, знакомы с парадигмой ООП, знать основные кострукции и типы, самостоятельно построить pipeline для обработки данных, создать свое небольшое flask приложение, знакомы с некоторыми сервисами AWS (ec2).
\end{abstract}

\section{Тематические блоки}

\subsection{Модуль 1}

\subsubsection{Неделя 1\\ Тема: Введение в Python}

\begin{itemize}
    \item  Особенности и возможности языка Python
    \item  Типы данных
\end{itemize}

\subsubsection{Неделя 2\\ Тема: }

\begin{itemize}
    \item 
    \item 
\end{itemize}

\subsubsection{Неделя 3\\ Тема: }

\begin{itemize}
    \item 
    \item 
\end{itemize}

\subsubsection{Неделя 4\\ Тема: }

\begin{itemize}
    \item 
    \item 
\end{itemize}

\subsubsection{Неделя 5\\ Тема: }

\begin{itemize}
    \item 
    \item 
\end{itemize}

\subsubsection{Неделя 6\\ Тема: }

\begin{itemize}
    \item 
    \item 
\end{itemize}

\subsubsection{Неделя 7\\ Тема: }

\begin{itemize}
    \item 
    \item 
\end{itemize}

\subsubsection{Личный проект 1\\ Тема: }


\subsection{Модуль 2}

\subsubsection{Неделя 8\\ Тема: }

\begin{itemize}
    \item 
    \item 
\end{itemize}

\subsubsection{Неделя 9\\ Тема: }

\begin{itemize}
    \item 
    \item 
\end{itemize}

\subsubsection{Неделя 10\\ Тема: }

\begin{itemize}
    \item 
    \item 
\end{itemize}

\subsubsection{Неделя 11\\ Тема: }

\begin{itemize}
    \item 
    \item 
\end{itemize}

\subsubsection{Неделя 12\\ Тема: }

\begin{itemize}
    \item 
    \item 
\end{itemize}

\subsubsection{Неделя 13\\ Тема: }

\begin{itemize}
    \item 
    \item 
\end{itemize}

\subsubsection{Неделя 14\\ Тема: }

\begin{itemize}
    \item 
    \item 
\end{itemize}

\subsubsection{Личный проект 2\\ Тема: }

\end{document}
